\chapter{Fondstyret}
\vspace{23pt}

Fondstyret styrer Onlines fond mellom generalforsamlingene. Fondstyrets medlemmer velges på generalforsamlingen og skal drive fondet mellom generalforsamlingene. For at Fondstyret skal være beslutningsdyktig må minst halvparten av representantene være tilstede.

Ingen kan inneha to verv i Fondstyret. Fondstyrets møter er lukket, men gjester kan inviteres dersom Fondstyret ønsker dette. Fondstyret skrives med stor ’F’ på samme måte som egennavn.

\section{Fondstyrets sammensetning}


Fondstyret består av:

\begin{liste}
  \item Økonomiansvarlig i Online
  \item Ridderne av det indre lager-medlem 1
  \item Ridderne av det indre lager-medlem 2
  \item Tidligere medlem fra Onlines Hovedstyre 1
  \item Tidligere medlem fra Onlines Hovedstyre 2
  \item Onlinemedlem 1
  \item Onlinemedlem 2
\end{liste}

Alle medlemmer av Fondstyret har stemmerett. Fondstyret velger sin egen leder. Ved stemmelikhet teller Fondstyret sin leder dobbelt.

\section{Valg av Fondstyre}


Fondstyret skal velges på generalforsamlingen, på samme måte som medlemmer til Onlines Hovedstyre som beskrevet i Onlines vedtekter av 3.6, med unntak av følgende:

I oddetallsår skal følgende styremedlemmer velges på nytt: Ridderne av det indre lager-medlem 1, tidligere medlem fra Onlines Hovedstyre 1, og Onlinemedlem 1.

I partallsår skal følgende styremedlemmer velges på nytt: Ridderne av det indre lager-medlem 2, tidligere medlem fra Onlines Hovedstyre 2, og Onlinemedlem 2.

\section{Krav til kandidater ved valg}
En kandidat til valg kan ikke være valgt til medlem av Onlines Hovedstyre. Avtroppende Hovedstyre kan stille til valg.

Kandidater for stillingene som tidligere medlem fra Onlines Hovedstyre kan ikke være medlem av Ridderne av det Indre Lager, eller tidligere medlem i Fondstyret.

Kandidater til stillingene som Onlinemedlem kan ikke være medlem av Ridderne av det Indre Lager, tidligere medlem av Onlines Hovedstyre, eller tidligere medlem i Fondstyret.

Minst et Onlinemedlem i Fondstyret skal ikke være nåværende eller tidligere medlem av en kjernekomité i Online.

\section{Trekke seg fra verv}

Dersom et medlem av Fondstyret trekker seg før perioden deres er over skal Fondstyret fylle den aktuelle stillingen med et medlem som oppfyller kravene i 4.3. Stillingen skal velges på nytt ved neste ordinære generalforsamling.

\section{Mislighold av verv}
Om et fondstyremedlem misligholder sine arbeidsoppgaver, kan ethvert medlem av Online stille mistillitsforslag overfor vedkommende. Mistillitsforslaget skal leveres skriftlig til Fondstyret, som skal behandle saken. Ved mistillitsforslag mot et fondstyremedlem blir den anklagede suspendert inntil Fondstyret har kommet med en avgjørelse. Mistillitsforslaget leses opp i Fondstyret, deretter skal den anklagede få en mulighet til å forsvare seg før Fondstyret diskuterer og avgjør saken uten den anklagede til stede. Dersom det stille mistillitsforslag til flere styremedlemmer av gangen skal medlemmene kalle inn til ekstraordinær generalforsamling etter 3.1.
