\chapter{generalforsamlingen}

Generalforsamlingen er fondets øverste organ og er uavhengig av gjeldende fondstyre. Generalforsamlingen avholdes samtidig med Onlines ordinære generalforsamling. For fondets generalforsamling gjelder følgende av Onlines vedtekter: 3.1, 3.4 og 3.5.
Den ordinære generalforsamlingen skal behandle egne punkter med årsberetning, innsendte saker, vedtektsendringer, valg, regnskap og budsjettforslag for neste budsjettperiode.
Alle medlemmer i Online har rett til å levere saks- og vedtektsforslag.

\section{Ekstraordinær generalforsamling}


Denne kan innkalles av Fondstyret, eller om minst det laveste av ti (10) eller 1/8 av medlemmene i Online ønsker det. Fristene for å kalle inn til ekstraordinær generalforsamling er halvert i forhold til fristene for ordinær, jamfør Onlines vedtekter 3.1.

Ekstraordinær generalforsamling skal kun den(de) saken(e) som står på dagsorden for den ekstraordinære generalforsamlingen.


\section{Organisering}


Ved generalforsamling er disse vervene nødvendig:
\begin{liste}
  \item Ordstyrer
  \item To referenter - skriver referat og samarbeider om renskriving
  \item Minst to til tellekorps - teller opp stemmene ved avstemming
  \item To paraferer - godkjenner referat fra generalforsmling og de endrede vedtekene i etterkant av generalforsamlingen.
\end{liste}

Ved generalforsamling som avholdes samtidig med generalforsamling i regi av Online, vil tilsvarende verv ved denne automatisk tilfalle disse vervene.

\section{Gjennomføring av valget}


Dersom det er mer enn en kandidat til et verv skal det avholdes anonymt valg for det aktuelle vervet. Man kan stemme “ingen” på valget. For å regnes som vinner av valget må en kandidat få minstover halvparten (1/2) av stemmene. Ved manglende flertall fjernes kandidaten med færrest stemmer og valget går inn i en ny runde.

Ved manglende flertall på kandidat med flest stemmer og stemmelikhet på de med færrest stemmer vil det avholdes en fullstendig ny runde.

Innehavere av verv sitter inntil endt generalforsamling hvor det er gjennomført et godkjent valg for det respektive vervet. Dersom generalforsamlingen ikke klarer å gjennomføre et valg må det kalles inn til ekstraordinær forsamling innen tre dager etter endt ordinær generalforsamling.

Ved opptelling av hemmelig valg skal tellekorps sitte i salen.

\subsection{Fraskrivelse av rett til å stille til valg}


Personer som er innstilt med et av følgende verv under Generalforsamlingen fraskriver seg sin rett til å stille til alle andre valg.
\begin{liste}
  \item Ordstyrer
  \item Tellekorps
\end{liste}

Med å stille til andre valg menes det at man ikke kan stille, eller bli nominert, til andre verv under Generalforsamlingen og valg til Fondstyret.
